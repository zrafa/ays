% Copyright 2004 by Till Tantau <tantau@users.sourceforge.net>.
%
% In principle, this file can be redistributed and/or modified under
% the terms of the GNU Public License, version 2.
%
% However, this file is supposed to be a template to be modified
% for your own needs. For this reason, if you use this file as a
% template and not specifically distribute it as part of a another
% package/program, I grant the extra permission to freely copy and
% modify this file as you see fit and even to delete this copyright
% notice. 

\documentclass{beamer}
% Replace the \documentclass declaration above
% with the following two lines to typeset your 
% lecture notes as a handout:
%\documentclass{article}
%\usepackage{beamerarticle}

\usepackage[utf8x]{inputenc}
\usepackage[spanish]{babel}
\usepackage{fancyvrb}
% \usepackage{verbatim}

% There are many different themes available for Beamer. A comprehensive
% list with examples is given here:
% http://deic.uab.es/~iblanes/beamer_gallery/index_by_theme.html
% You can uncomment the themes below if you would like to use a different
% one:
%\usetheme{AnnArbor}
%\usetheme{Antibes}
%\usetheme{Bergen}
%\usetheme{Berkeley}
%\usetheme{Berlin}
%\usetheme{Boadilla}
%\usetheme{boxes}
%\usetheme{CambridgeUS}
%\usetheme{Copenhagen}
%\usetheme{Darmstadt}
%\usetheme{default}
%\usetheme{Frankfurt}
%  \usetheme{Goettingen}
%\usetheme{Hannover}
%\usetheme{Ilmenau}
%\usetheme{JuanLesPins}
%\usetheme{Luebeck}
%\usetheme{Madrid}
%\usetheme{Malmoe}
%\usetheme{Marburg}
%\usetheme{Montpellier}
%\usetheme{PaloAlto}
%\usetheme{Pittsburgh}
%\usetheme{Rochester}
\usetheme{Singapore}
%\usetheme{Szeged}
%\usetheme{Warsaw}


\title{Automatización y Scripting}


\begin{document}

\begin{frame}
  \titlepage

\end{frame}






\begin{frame}{}
\frametitle{Órdenes Compuestas}
Las órdenes compuestas más utilizadas son las siguientes :
\begin{itemize}
\item \textbf{Las que utilizan palabras reservadas : } if, for, select, case, while until, function.
\item \textbf{Una (lista) }: lista  se  ejecuta  en  un subshell. Después de que la orden se completa, las asignaciones a variables y órdenes internas que afectaran al
              entorno del shell no permanecen en efecto.
\texttt{ Ejemplo :  (ls /etc ; A=2; ls /root;) }
\item \textbf{Una \{ lista; \} }: 
              lista se ejecuta simplemente en el entorno del shell en curso.  lista debe terminarse con un salto de línea o un punto y  coma.
\texttt{ Ejemplo :  \{ ls /etc ; A=2; ls /root; \} }
\item
\textbf{Una ((expresión)) } La expresión se evalúa de acuerdo a las reglas de EVALUACIÓN ARITMÉTICA.  
\item
\textbf{Una [[ expresión ]] } Devuelve un estado de 0 ó 1 dependiendo de la EVALUACIÓN CONDICIONAL. 
\end{itemize}
\end{frame}{}



\begin{Verbatim}


Evaluaciones Aritmeticas más utilizadas :


       - +    menos y más unarios
       **     exponenciación
       * / %  multiplicación, división, resto
       + -    adición, sustracción


Ejemplo :
	x=2
	y=3
	rectangulo=$((x*2 + y*2))
\end{Verbatim}


\begin{frame}
\frametitle{Expresiones Condicionales}
Las expresiones condicionales son empleadas por la orden compuesta [[ y por las órdenes internas test. Muy utilizadas con \texttt{if}. Las mas utilizadas son :

\begin{itemize}
\item \textbf{-a nombre} Verdadero si nombre existe y es un archivo.
\item \textbf{-d nombre} Verdadero si nombre existe y es un directorio.
\item \textbf{-f nombre} Verdadero si nombre existe y es un archivo común.
\item \textbf{cad1 == cad2 } Verdadero si las cadenas son iguales.
\item \textbf{cad1 != cad2 } Verdadero si las cadenas son distintas.
\item \textbf{arg1 -eq arg2} Aritmético. Verdadero si arg1 es igual a arg2.
\item \textbf{arg1 -lt arg2} Aritmético. Verdadero si arg1 es menor a arg2.
\item \textbf{arg1 -gt arg2} Aritmético. Verdadero si arg1 es mayor a arg2.

\end{itemize}
\end{frame}

\begin{Verbatim}

Ejemplos de expresiones condicionales para analizar :

 N=`wc -l /etc/passwd`
 echo $N
 if [[ "$N" -gt "20" ]] ; then 
	echo "Demasiados usuarios"
 fi

 SEMANA=$(date +"%V")
 if [[ $SEMANA -gt "10" ]]; then
	echo "Mas de 10 semanas, parcial seguro."
 fi

 YO=$(whoami)
 if [[ "$(YO)" != 'root' && "$(YO)" != 'sysadmin' ]]; then
        echo "Soy un tipo comun.. (Autor: de Narvaez)"
        exit 1;
 fi

\end{Verbatim}


\begin{Verbatim}
Crear el siguiente script con nombre /tmp/f.sh
Ejecutar : /tmp/f.sh

#!/bin/bash
# Este script informa detalles de un archivo.

FILENAME="$1"
echo "Características de $FILENAME:"

if [ -f $FILENAME ]; then
  echo "Tamanio: $(ls -lh $FILENAME | awk '{ print $5 }')"
  echo "Tipo: $(file $FILENAME | cut -d":" -f2 -)"
  echo "Inodo: $(ls -i $FILENAME | cut -d" " -f1 -)"
  echo "$(df -h $FILENAME | tail -1 | 
awk '{ print "On",$1", \
el cual esta montado como la ",$6,"particion."}')"
fi
\end{Verbatim}



\begin{frame}
\frametitle{Funciones - Argumentos}
       Una  función define una serie de órdenes para ser ejecutadas bajo un nombre. Características

\begin{itemize}
\item No  se crea ningún nuevo proceso para ejecutar la función.
\item Lo argumentos de  la  función  se
       convierten  en los parámetros posicionales durante su ejecución.
\item 
       Variables  locales  a la función se pueden declarar con la orden interna local.  
Normalmente, las
       variables y sus valores se comparten entre la función y quien la llama, como variables globales.

\item
       Si se ejecuta la orden interna return en una función, éste finaliza y la ejecución se reanuda con
       la  siguiente  orden tras la llamada a la función. 

\end{itemize}
\end{frame}

\begin{frame}
\frametitle{Funciones - Argumentos - Continuación}
\begin{itemize}
\item La definición de una función es con \texttt{function nombre() \{ comandos; \} }
\item Los argumentos se convierten en parámetros posicionales con los nombres \$1, \$2, ..., \$n
\item Los argumentos es información que puede ser enviada a la función en forma de una lista de cadenas separadas por espacio.\\ 
\item Los parámetros posicionales son variables, por lo que pueden ser utilizadas como las definidas explicitamente.
\texttt{Ejemplo : function sumar()\{ echo \$((\$1+\$2+\$3)); \} }\\ 
\texttt{Ejemplo de llamada : sumar 2 3 4 }
\end{itemize}
\end{frame}






\begin{frame}
Bibliografía : 
\begin{itemize}
\item man bash
\item Libro Bash Guide for Beginners, de Machtelt Garrels 
\item Libro Advanced Bash-Scripting Guide, de Mendel Cooper
\end{itemize}

Ambos libros están disponibles en varios formatos diferentes en :

Bibliografía suplementaria sugerida : 
\begin{itemize}
\item Libro El Entorno De Programación Unix, de Kernighan, y Rob Pike
\end{itemize}

\end{frame}



\end{document}


