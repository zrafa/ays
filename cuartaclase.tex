% Copyright 2004 by Till Tantau <tantau@users.sourceforge.net>.
%
% In principle, this file can be redistributed and/or modified under
% the terms of the GNU Public License, version 2.
%
% However, this file is supposed to be a template to be modified
% for your own needs. For this reason, if you use this file as a
% template and not specifically distribute it as part of a another
% package/program, I grant the extra permission to freely copy and
% modify this file as you see fit and even to delete this copyright
% notice. 

\documentclass{beamer}
% Replace the \documentclass declaration above
% with the following two lines to typeset your 
% lecture notes as a handout:
%\documentclass{article}
%\usepackage{beamerarticle}

\usepackage[utf8x]{inputenc}
\usepackage[spanish]{babel}
\usepackage{fancyvrb}
% \usepackage{verbatim}

% There are many different themes available for Beamer. A comprehensive
% list with examples is given here:
% http://deic.uab.es/~iblanes/beamer_gallery/index_by_theme.html
% You can uncomment the themes below if you would like to use a different
% one:
%\usetheme{AnnArbor}
%\usetheme{Antibes}
%\usetheme{Bergen}
%\usetheme{Berkeley}
%\usetheme{Berlin}
%\usetheme{Boadilla}
%\usetheme{boxes}
%\usetheme{CambridgeUS}
%\usetheme{Copenhagen}
%\usetheme{Darmstadt}
%\usetheme{default}
%\usetheme{Frankfurt}
%  \usetheme{Goettingen}
%\usetheme{Hannover}
%\usetheme{Ilmenau}
%\usetheme{JuanLesPins}
%\usetheme{Luebeck}
%\usetheme{Madrid}
%\usetheme{Malmoe}
%\usetheme{Marburg}
%\usetheme{Montpellier}
%\usetheme{PaloAlto}
%\usetheme{Pittsburgh}
%\usetheme{Rochester}
\usetheme{Singapore}
%\usetheme{Szeged}
%\usetheme{Warsaw}


\title{Automatización y Scripting}


\begin{document}

\begin{frame}
  \titlepage

\end{frame}





\begin{Verbatim}

# Variables locales y su alcance

$ TIERRA=mundo	# definición de una variable local
$ echo $TIERRA	# muestra el contenido de TIERRA
mundo
$ bash		# se inicia un nuevo shell hijo
$ echo $TIERRA	# la variable TIERRA está vacía!
$ exit		# se vuelve al shell anterior
$ echo $TIERRA	# la variable TIERRA aún existe acá
mundo
$ export TIERRA
$ bash		# se inicia un nuevo shell hijo
$ echo $TIERRA	# la variable TIERRA es vista!
mundo
$ EARTH=$TIERRA	# definimos una nueva variable
$ echo $EARTH	# muestra el contenido de EARTH
mundo
\end{Verbatim}

\begin{frame}{}
\frametitle{Órdenes Compuestas}
Las órdenes compuestas más utilizadas son las siguientes :
\begin{itemize}
\item \textbf{Las que utilizan palabras reservadas : } if, for, select, case, while until, function.
\item \textbf{Una (lista) }: lista  se  ejecuta  en  un subshell. Después de que la orden se completa, las asignaciones a variables y órdenes internas que afectaran al
              entorno del shell no permanecen en efecto.
\texttt{ Ejemplo :  (ls /etc ; A=2; ls /root;) }
\item \textbf{Una \{ lista; \} }: 
              lista se ejecuta simplemente en el entorno del shell en curso.  lista debe terminarse con un salto de línea o un punto y  coma.
\texttt{ Ejemplo :  \{ ls /etc ; A=2; ls /root; \} }
\item
\textbf{Una ((expresión)) } La expresión se evalúa de acuerdo a las reglas de EVALUACIÓN ARITMÉTICA.  
\item
\textbf{Una [[ expresión ]] } Devuelve un estado de 0 ó 1 dependiendo de la EVALUACIÓN CONDICIONAL. 
\end{itemize}
\end{frame}{}



\begin{Verbatim}


Evaluaciones Aritmeticas más utilizadas :


       - +    menos y más unarios
       **     exponenciación
       * / %  multiplicación, división, resto
       + -    adición, sustracción


Ejemplo :
	x=2
	y=3
	rectangulo=$((x*2 + y*2))
\end{Verbatim}


\begin{frame}
\frametitle{Expresiones Condicionales}
Las expresiones condicionales son empleadas por la orden compuesta [[ y por las órdenes internas test. Muy utilizadas con \texttt{if}. Las mas utilizadas son :

\begin{itemize}
\item \textbf{-a nombre} Verdadero si nombre existe y es un archivo.
\item \textbf{-d nombre} Verdadero si nombre existe y es un directorio.
\item \textbf{-f nombre} Verdadero si nombre existe y es un archivo común.
\item \textbf{cad1 == cad2 } Verdadero si las cadenas son iguales.
\item \textbf{cad1 != cad2 } Verdadero si las cadenas son distintas.
\item \textbf{arg1 -eq arg2} Aritmético. Verdadero si arg1 es igual a arg2.
\item \textbf{arg1 -lt arg2} Aritmético. Verdadero si arg1 es menor a arg2.
\item \textbf{arg1 -gt arg2} Aritmético. Verdadero si arg1 es mayor a arg2.

\end{itemize}
\end{frame}

\begin{Verbatim}

Ejemplos de expresiones condicionales para analizar :

 N=`wc -l /etc/passwd`
 echo $N
 if [[ "$N" -gt "20" ]] ; then echo "Demasiados usuarios" ; fi

 SEMANA=$(date +"%V")
 if [[ $SEMANA -gt "10" ]]; then
	echo "Ya van mas de 10 semanas, se viene el parcial seguro."
 fi

 if [[ "$(whoami)" != 'root' ]]; then
        echo "Soy un tipo comun.. (Autor: Alumno F. de Narvaez)"
        exit 1;
 fi


A=/etc/shells
B=`ls /etc/she*`

if [[ "$A" == "$B" ]] ; then echo "A y B son iguales" ; fi

if [[ "$A" != "$B" ]] ; then
	echo "A y B son distintos"
else
	echo "A y B son iguales"
fi

\end{Verbatim}




\begin{Verbatim}

################################################
#
# Finalizando Hoy - Comandos Útiles
# Trabajo Practico 
# Analizar y explicar que sucede en cada caso :

ls -l / | tail 
ls -l / | head 
ls -l / | head -12

ls -l / | awk '{print $9}'
ls -l / | awk '{print $5}'
ls -l / | awk '{print $5" "$9}'
ls -l / | awk '{print $5" "$9}' | sort -n

ls /lib
ls /lib | grep lib
ls /lib | grep -v lib
\end{Verbatim}




\begin{frame}
Bibliografía : 
\begin{itemize}
\item man bash
\item Libro Bash Guide for Beginners, de Machtelt Garrels 
\item Libro Advanced Bash-Scripting Guide, de Mendel Cooper
\end{itemize}

Ambos libros están disponibles en varios formatos diferentes en :

Bibliografía suplementaria sugerida : 
\begin{itemize}
\item Libro El Entorno De Programación Unix, de Kernighan, y Rob Pike
\end{itemize}

\end{frame}



\end{document}


