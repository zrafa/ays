% Copyright 2004 by Till Tantau <tantau@users.sourceforge.net>.
%
% In principle, this file can be redistributed and/or modified under
% the terms of the GNU Public License, version 2.
%
% However, this file is supposed to be a template to be modified
% for your own needs. For this reason, if you use this file as a
% template and not specifically distribute it as part of a another
% package/program, I grant the extra permission to freely copy and
% modify this file as you see fit and even to delete this copyright
% notice. 

\documentclass{beamer}
% Replace the \documentclass declaration above
% with the following two lines to typeset your 
% lecture notes as a handout:
%\documentclass{article}
%\usepackage{beamerarticle}

\usepackage[utf8x]{inputenc}
\usepackage[spanish]{babel}
\usepackage{verbatim}

% There are many different themes available for Beamer. A comprehensive
% list with examples is given here:
% http://deic.uab.es/~iblanes/beamer_gallery/index_by_theme.html
% You can uncomment the themes below if you would like to use a different
% one:
%\usetheme{AnnArbor}
%\usetheme{Antibes}
%\usetheme{Bergen}
%\usetheme{Berkeley}
%\usetheme{Berlin}
%\usetheme{Boadilla}
%\usetheme{boxes}
%\usetheme{CambridgeUS}
%\usetheme{Copenhagen}
%\usetheme{Darmstadt}
%\usetheme{default}
%\usetheme{Frankfurt}
%  \usetheme{Goettingen}
%\usetheme{Hannover}
%\usetheme{Ilmenau}
%\usetheme{JuanLesPins}
%\usetheme{Luebeck}
%\usetheme{Madrid}
%\usetheme{Malmoe}
%\usetheme{Marburg}
%\usetheme{Montpellier}
%\usetheme{PaloAlto}
%\usetheme{Pittsburgh}
%\usetheme{Rochester}
\usetheme{Singapore}
%\usetheme{Szeged}
%\usetheme{Warsaw}

\title{Automatización y Scripting}

% A subtitle is optional and this may be deleted
% \subtitle{\tiny Tecnicatura Universitaria en Administración de Sistemas y Software Libre}

% \author{F.~Author\inst{1} \and S.~Another\inst{2}}
\author{Docentes: Rafael Ignacio Zurita rafa@fi.uncoma.edu.ar \\
	Miriam Lechner mtl@fi.uncoma.edu.ar }
% - Give the names in the same order as the appear in the paper.
% - Use the \inst{?} command only if the authors have different
%   affiliation.

\institute[Universities of Somewhere and Elsewhere] % (optional, but mostly needed)
{
 % \inst{2}%
Tecnicatura Universitaria en Administración de Sistemas y Software Libre\\
Departamento Ingeniería de Computadoras\\
Universidad Nacional del Comahue}
% - Use the \inst command only if there are several affiliations.
% - Keep it simple, no one is interested in your street address.

\date{Primer Cuatrimestre de 2014}
% - Either use conference name or its abbreviation.
% - Not really informative to the audience, more for people (including
%   yourself) who are reading the slides online

\subject{Theoretical Computer Science}
% This is only inserted into the PDF information catalog. Can be left
% out. 

% If you have a file called "university-logo-filename.xxx", where xxx
% is a graphic format that can be processed by latex or pdflatex,
% resp., then you can add a logo as follows:

% \pgfdeclareimage[height=0.5cm]{university-logo}{university-logo-filename}
% \logo{\pgfuseimage{university-logo}}

% Delete this, if you do not want the table of contents to pop up at
% the beginning of each subsection:
% \AtBeginSubsection[]
% {
%   \begin{frame}<beamer>{Outline}
%     \tableofcontents[currentsection,currentsubsection]
%   \end{frame}
% }

% Let's get started
\begin{document}

\begin{frame}
  \titlepage
\end{frame}






%  Contenido
%  Tema
%  Marco 
%  Historia
%  Ejercicio
%  Caracteristicas
%  Ejemplo de un programa risc mips
%  Ejemplo de un programa x86


\begin{frame}{}
Requisitos : 
\begin{itemize}
\item Introducción a la Administración de Sistemas.
\item Administración de Sistemas.
\end{itemize}

\end{frame}


\begin{frame}{}
Horario : 
\begin{itemize}
\item Viernes de 19hs a 22hs.
\end{itemize}

Mucha práctica en casa!

\end{frame}


\begin{frame}{}
Objetivos: 
\begin{itemize}
\item Conocer principios generales de la programación de scripting.
\item Automatización de tareas de administración utilizando scripts.
\end{itemize}

Contenidos mínimos:

\begin{itemize}
\item Técnicas y principios de scripting.
\item Comandos Unix/Linux utiles. 
\item Programación con scripts para la automatización de tareas.
\end{itemize}

\end{frame}


\begin{frame}{}
Requisitos para aprobar la materia : 
\begin{itemize}
\item 13 de Junio : PARCIAL; o
\item 27 de Junio : RECUPERATORIO DEL PARCIAL
\item Un trabajo práctico a entregar durante el cursado (antes del parcial)
\end{itemize}

Dias sin clase : 
\begin{itemize}
\item 18 de Abril FERIADO : Viernes Santo
\item 2 de Mayo FERIADO : Puente Turistico
\item 20 de Junio FERIADO : Día Paso a la Inmortalidad del Gral. Manuel Belgrano
\end{itemize}
Cantidad total de clases : 11.
\end{frame}

\begin{frame}{}
Bibliografía : 
\begin{itemize}
\item man bash
\item Libro Bash Guide for Beginners, de Machtelt Garrels 
\item Libro Advanced Bash-Scripting Guide, de Mendel Cooper
\end{itemize}

Ambos libros están disponibles en varios formatos diferentes en :

Bibliografía suplementaria sugerida : 
\begin{itemize}
\item Libro El Entorno De Programacion Unix, de Kernighan, y Rob Pike
\end{itemize}

\end{frame}



\end{document}


