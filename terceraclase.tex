% Copyright 2004 by Till Tantau <tantau@users.sourceforge.net>.
%
% In principle, this file can be redistributed and/or modified under
% the terms of the GNU Public License, version 2.
%
% However, this file is supposed to be a template to be modified
% for your own needs. For this reason, if you use this file as a
% template and not specifically distribute it as part of a another
% package/program, I grant the extra permission to freely copy and
% modify this file as you see fit and even to delete this copyright
% notice. 

\documentclass{beamer}
% Replace the \documentclass declaration above
% with the following two lines to typeset your 
% lecture notes as a handout:
%\documentclass{article}
%\usepackage{beamerarticle}

\usepackage[utf8x]{inputenc}
\usepackage[spanish]{babel}
\usepackage{fancyvrb}
% \usepackage{verbatim}

% There are many different themes available for Beamer. A comprehensive
% list with examples is given here:
% http://deic.uab.es/~iblanes/beamer_gallery/index_by_theme.html
% You can uncomment the themes below if you would like to use a different
% one:
%\usetheme{AnnArbor}
%\usetheme{Antibes}
%\usetheme{Bergen}
%\usetheme{Berkeley}
%\usetheme{Berlin}
%\usetheme{Boadilla}
%\usetheme{boxes}
%\usetheme{CambridgeUS}
%\usetheme{Copenhagen}
%\usetheme{Darmstadt}
%\usetheme{default}
%\usetheme{Frankfurt}
%  \usetheme{Goettingen}
%\usetheme{Hannover}
%\usetheme{Ilmenau}
%\usetheme{JuanLesPins}
%\usetheme{Luebeck}
%\usetheme{Madrid}
%\usetheme{Malmoe}
%\usetheme{Marburg}
%\usetheme{Montpellier}
%\usetheme{PaloAlto}
%\usetheme{Pittsburgh}
%\usetheme{Rochester}
\usetheme{Singapore}
%\usetheme{Szeged}
%\usetheme{Warsaw}


\title{Automatización y Scripting}


\begin{document}

\begin{frame}
  \titlepage

\end{frame}





\begin{frame}{}
\frametitle{Variables}
\begin{itemize}
\item Una variable es una marca/etiqueta/identificador/o nombre en el programa, y refiere a una ubicacion de memoria.

\item Algunas variables pueden son definidas por el programador, otras son variables especiales (por ejemplo PATH que es una variable de ambiente).
\item Las nombres de las variables diferencian las mayúsculas de las minúsculas.
\item Los nombres deben comenzar con una letra o guíon bajo \_.
\item El shell de UNIX interpreta los comandos del usuario. Los comandos pueden ser :
\end{itemize}

\end{frame}


\begin{frame}{}
\frametitle{Variables - Continuacion}
Los shell UNIX más avanzados soportan cuatro tipos de datos:
\begin{itemize}
\item Cadenas de caracteres (strings), que son secuencias de caracteres alfanuméricos. 
\item Enteros (Integers), los cuales son números enteros.
\item Punto Flotante (Floats), que son números decimales en punto flotante. 
\item Arreglos (Arays), que son secuencias de variables.
\end{itemize}
\textbf{BASH utiliza cadenas de caracteres y enteros}
\end{frame}{}

\begin{frame}{}
\frametitle{Usando Varibables}
\begin{itemize}
\item       A  una  variable se le puede asignar un valor mediante una sentencia de
       la forma :\\ 
             \texttt{        nombre=[valor]}
\item Para hacer referencia al valor de una variable se utiliza el signo pesos "\$". Ej.:\\
             \texttt{        echo \$HOME}
\item Puede utilizar el símbolo \$ para asignar el valor de una variable a otra variable. EJ.:\\
             \texttt{        casa=\$HOME}
\item Para eliminar una variable se utiliza el comando \texttt{unset}. 

\end{itemize}
\end{frame}{}


\begin{frame}{}
\frametitle{Variables locales y su alcance}
\begin{itemize}
\item El alcance de una variable local está destinado al shell en donde se ha creado la variable. Los procesos hijos no pueden acceder a las variables locales del padre.
\item Una variable puede ser agregada al entorno para ser vista por los procesos hijoas mediante la orden \texttt{export}
\end{itemize}
\end{frame}{}

\begin{Verbatim}

# Variables locales y su alcance

$ TIERRA=mundo	# definicion de una variable local
$ echo $TIERRA	# muestra el contenido de TIERRA
mundo
$ bash		# se inicia un nuevo shell hijo
$ echo $TIERRA	# la variable TIERRA está vacía!
$ exit		# se vuelve al shell anterior
$ echo $TIERRA	# la variable TIERRA aún existe acá
mundo
$ export TIERRA
$ bash		# se inicia un nuevo shell hijo
$ echo $TIERRA	# la variable TIERRA es vista!
mundo
$ EARTH=$TIERRA	# definimos una nueva variable
$ echo $EARTH	# muestra el contenido de EARTH
mundo
\end{Verbatim}

\begin{frame}{}
\frametitle{Comillas}
% Las comillas se utilizan para evitar que el shell procese espacios y caracteres especiales.
\begin{itemize}
\item Las comillas simples (') permiten ocultar el significado de todos los caracteres 
		especiales, o metacaracteres. Incluyendo comillas dobles, 
		invertidas, espacios y saltos de linea. Por este motivo evitan 
		la sustitución de variables y comandos.
\item Las comillas dobles (") permiten ocultar el significado de los caracteres especiales, 
 		espacios y saltos de linea con excepción de las comillas 
 		invertidas, y los caracteres \$ y barra invertida. Permiten la expansión de variables y comandos.
\item
Las Comillas invertidas (`) permiten substituir órdenes con su resultado.
		Cuando encerramos un comando entre comillas invertidas, el shell 
ejecuta y 
		sustituye el comando y las comillas invertidas por el resultado del 
		la ejecución.
\end{itemize}
\end{frame}{}


\begin{Verbatim}

# Comillas - Verificar : 

$ echo '* y $HOME'	# comillas simples
$ echo * y $HOME

$ HOGARYRUTA="$HOME y $PATH"	# comillas dobles
$ echo $HOGARYRUTA
$ NUEVOHOGAR=$HOME y $PATH


$ NOMBRE=`hostname`	# comillas invertidas
$ echo $NOMBRE
$ echo Los siguientes usuarios están online: `who` 

# Puede utilizar tambien $(hostame) en lugar de
# las comillas invertidas!


\end{Verbatim}


\begin{Verbatim}

# Comillas - Analizar :




$ precio='$30'
$ echo $precio 
$30 
$ item=helado 
$ linea="El $item vale $precio" 
$ echo $linea
El helado vale $30


\end{Verbatim}




\begin{frame}
\frametitle{Palabras Reservadas}
 \begin{itemize}
       \item Son palabras que  tienen  un  significado  especial
       para  el  shell :\\

\texttt{       ! case  do done elif else esac fi for function if in select then  until while \{ \} time [[ ]]}\\

\item Permiten crear órdenes compuestas en forma de bloques de código, órdenes condicionales y repetitivas.
\end{itemize}
\end{frame}

\begin{frame}
\frametitle{Palabras Reservadas - for - Repetitiva}
 \begin{itemize}
\item \texttt{for nombre [ in palabra; ] do lista ; done}

\item La lista de palabras que va detrás de in se  expande,  generando
              una  lista  de elementos. La variable nombre se define como cada
              elemento de la lista en cada iteración, y lista se ejecuta  cada
              vez. Ejemplos :\\


\texttt{       Ej. 1: for i in 1 2 3 4 5 ; do echo \$i ; done}\\
\texttt{       Ej. 2: for i in `ls /`; do echo \$i ; done}\\
\texttt{       Ej. 3: }\\
\texttt{       for i in 1 2 3 `ls /`; do }\\ \ \ \
\texttt{           echo \$i}\\
\texttt{       done}

\end{itemize}
\end{frame}

\begin{frame}
\frametitle{Palabras Reservadas - if - Condicional}
 \begin{itemize}
\item \texttt{ if  lista;  then lista; [ elif lista; then lista; ] ... [ else lista; ] fi }
\item 
              La lista if se ejecuta. Si su estado de salida es cero, se  ejecuta  la  lista  then.  De  otro modo, se ejecuta por turno cada
              lista elif, y si su estado de salida  es  cero,  se  ejecuta  la
              lista  then  correspondiente  y  la orden se completa. Si no, se
              ejecuta la lista else si está presente. Ejemplos :\\

\end{itemize}
\end{frame}

\begin{Verbatim}

# IF - Analizar :




A=/etc/shells
B=`ls /etc/she*`

if [[ "$A" == "$B" ]] ; then echo "A y B son iguales" ; fi

if [[ "$A" != "$B" ]] ; then
	echo "A y B son distintos"
else
	echo "A y B son iguales"
fi

\end{Verbatim}




\begin{Verbatim}

################################################
#
# Finalizando Hoy - Comandos Utiles
#
# Trabajo Practico 
# Analizar y explicar que sucede en cada caso :

ls -l / | tail 
ls -l / | head 
ls -l / | head -12

ls -l / | awk '{print $9}'
ls -l / | awk '{print $5}'
ls -l / | awk '{print $5" "$9}'
ls -l / | awk '{print $5" "$9}' | sort -n

ls /lib
ls /lib | grep lib
ls /lib | grep -v lib
\end{Verbatim}




\begin{frame}
Bibliografía : 
\begin{itemize}
\item man bash
\item Libro Bash Guide for Beginners, de Machtelt Garrels 
\item Libro Advanced Bash-Scripting Guide, de Mendel Cooper
\end{itemize}

Ambos libros están disponibles en varios formatos diferentes en :

Bibliografía suplementaria sugerida : 
\begin{itemize}
\item Libro El Entorno De Programacion Unix, de Kernighan, y Rob Pike
\end{itemize}

\end{frame}



\end{document}


