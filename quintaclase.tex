% Copyright 2004 by Till Tantau <tantau@users.sourceforge.net>.
%
% In principle, this file can be redistributed and/or modified under
% the terms of the GNU Public License, version 2.
%
% However, this file is supposed to be a template to be modified
% for your own needs. For this reason, if you use this file as a
% template and not specifically distribute it as part of a another
% package/program, I grant the extra permission to freely copy and
% modify this file as you see fit and even to delete this copyright
% notice. 

\documentclass{beamer}
% Replace the \documentclass declaration above
% with the following two lines to typeset your 
% lecture notes as a handout:
%\documentclass{article}
%\usepackage{beamerarticle}

\usepackage[utf8x]{inputenc}
\usepackage[spanish]{babel}
\usepackage{fancyvrb}
% \usepackage{verbatim}

% There are many different themes available for Beamer. A comprehensive
% list with examples is given here:
% http://deic.uab.es/~iblanes/beamer_gallery/index_by_theme.html
% You can uncomment the themes below if you would like to use a different
% one:
%\usetheme{AnnArbor}
%\usetheme{Antibes}
%\usetheme{Bergen}
%\usetheme{Berkeley}
%\usetheme{Berlin}
%\usetheme{Boadilla}
%\usetheme{boxes}
%\usetheme{CambridgeUS}
%\usetheme{Copenhagen}
%\usetheme{Darmstadt}
%\usetheme{default}
%\usetheme{Frankfurt}
%  \usetheme{Goettingen}
%\usetheme{Hannover}
%\usetheme{Ilmenau}
%\usetheme{JuanLesPins}
%\usetheme{Luebeck}
%\usetheme{Madrid}
%\usetheme{Malmoe}
%\usetheme{Marburg}
%\usetheme{Montpellier}
%\usetheme{PaloAlto}
%\usetheme{Pittsburgh}
%\usetheme{Rochester}
\usetheme{Singapore}
%\usetheme{Szeged}
%\usetheme{Warsaw}


\title{Automatización y Scripting}


\begin{document}

\begin{frame}
  \titlepage

\end{frame}






\begin{frame}{}
\frametitle{La Clase de los 10 Ejemplos}
\begin{enumerate}
\item Usando parametros posicionales
\item Usando parametros posicionales
\end{enumerate}
\end{frame}{}



\begin{Verbatim}
#!/bin/bash

# Ejemplo 1. Utilizacion de parametros posicionales
# "$*"  "$1c$2c$3c…c${N}"
# "$@"  "$1" "$2" "$3" … "${N}"
# $#    Indica la cantidad de argumentos

if [[ "$#" -lt 1 ]] ; then
        echo "Error. Necesita al menos un argumento"
        exit 1
fi

for i in "$*" ; do
        echo $i
done

for i in "$@" ; do
        echo $i
done
\end{Verbatim}


\begin{Verbatim}


Ejemplo 1. Utilizacion de parametros posicionales

- Ejecutar una vez sin argumentos

- Ejecutar una segunda vez con 5 argumentos.
  Ejemplo : 

# ./ej1.sh lunes martes miercoles 4 5
lunes martes miercoles 4 5
lunes
martes
miercoles
4
5


\end{Verbatim}



\begin{Verbatim}
#!/bin/bash

# Ejemplo 2. Utilizacion de case-esac para seleccion
#  case palabra in [ ( patrón [ | patrón ] ... ) lista ;; ] ... esac

if [[ "$#" -lt 2 ]] ; then
        echo "Uso: $0 Signalnumber PID"; exit 1
fi

case "$1" in
        9) echo  "Enviando SIGKILL al proceso $2"
           kill -SIGKILL $2
           ;;
        15)  echo  "Enviando SIGTERM al proceso $2"
            kill -SIGTERM $2
            ;;
        *) echo "$1 no es controlada en este programa"
           ;;
esac
\end{Verbatim}



\begin{Verbatim}

Ejemplo 2. Utilizacion de case-esac para seleccion

- Ejecutar una vez sin argumentos

- Ejecutar una segunda vez con 2 argumentos.
  El primer argumento es el numero de senial a enviar.
  El segundo argumento es el numero de proceso.
  Ejemplo : 

# ./ej2.sh 15 13425
Enviando SIGTERM al proceso 13425






\end{Verbatim}

\begin{Verbatim}
#!/bin/bash

# Ejemplo 3. read para interactuar con el usuario
# read [nombre..] lee  una  línea  desde  la  entrada
# estándar, y la primera palabra se asigna al  primer  nombre,

N=0

printf "Ingrese cantidad de lineas : "
read NRO_LINEAS

cat /etc/passwd | 
while read linea && [[ "$N" -lt NRO_LINEAS ]] ; do
        echo "Linea $N : " $linea

        ((N+=1))
done

\end{Verbatim}


\begin{Verbatim}
#!/bin/bash

# Ejemplo 4. Programas elaborados. Programas con opciones
# Como controlar diferentes opciones.
# Uso de while, case, shift y break.

while :
do
    case "$1" in
      -h | --help)
          echo "Uso: ej4.sh [-h] [--info paq] [--buscar paq]"
          exit 0
          ;;
      --info)
          apt-cache show $2
          shift 2
          ;;




      --buscar)
          apt-cache search $2 | grep $2
          shift 2
          ;;
      -*)
          echo "Error: No reconozco el argumento: $1"
          exit 1
          ;;
      *)  # No hay mas argumentos
          break
          ;;
    esac
done



\end{Verbatim}

\begin{Verbatim}
#!/bin/bash

# Ejemplo 5. Comunicandonos con el mundo.
# Enviando correos al administrador

# Este programa realiza un backup e informa por email

TEMP=/tmp/tmp.$$


echo "To: autoyscripting@gmail.com
From: autoyscripting@gmail.com
Subject: Informacion del BACKUP

" >$TEMP

tar czf /tmp/backup-etc.tar.gz /etc  >>$TEMP  2>&1






if [[ "$?" == "0" ]]; then
        echo "El backup fue exitoso" >>$TEMP
else
        echo "El backup tuvo problemas. Revisar." >>$TEMP
fi

cat $TEMP | msmtp -a default  autoyscripting@gmail.com
rm $TEMP

\end{Verbatim}



 fi

 SEMANA=$(date +"%V")
 if [[ $SEMANA -gt "10" ]]; then
	echo "Mas de 10 semanas, parcial seguro."
 fi

 YO=$(whoami)
 if [[ "$YO" != 'root' && "$YO" != 'sysadmin' ]]; then
        echo "Soy un tipo comun.. (Autor: de Narvaez)"
        exit 1;
 fi

\end{Verbatim}


\begin{Verbatim}
Crear el siguiente script con nombre /tmp/f.sh
Ejecutar : /tmp/f.sh

#!/bin/bash
# Este script informa detalles de un archivo.

FILENAME="$1"
echo "Características de $FILENAME:"

if [ -f $FILENAME ]; then
  echo "Tamanio: $(ls -lh $FILENAME | awk '{ print $5 }')"
  echo "Tipo: $(file $FILENAME | cut -d":" -f2 -)"
  echo "Inodo: $(ls -i $FILENAME | cut -d" " -f1 -)"
  echo "$(df -h $FILENAME | tail -1 | 
  awk '{ print "En",$1", \
  el cual esta montado como la ",$6,"particion."}')"
fi
\end{Verbatim}



\begin{frame}
\frametitle{Funciones - Argumentos}
       Una  función define una serie de órdenes para ser ejecutadas bajo un nombre. Características

\begin{itemize}
\item No  se crea ningún nuevo proceso para ejecutar la función.
\item Lo argumentos de  la  función  se
       convierten  en los parámetros posicionales durante su ejecución.
\item 
       Variables locales a la función se pueden declarar con la orden interna \texttt{local}.
Normalmente, las
       variables y sus valores se comparten entre la función y quien la llama, como variables globales.

\item
       Si se ejecuta la orden interna \texttt{return} la función finaliza y la ejecución se reanuda con
       la  siguiente  orden tras la llamada a la función. 

\end{itemize}
\end{frame}

\begin{frame}
\frametitle{Funciones - Argumentos - Continuación}
\begin{itemize}
\item La definición de una función es con \texttt{function nombre() \{ comandos; \} }
\item Los argumentos se convierten en parámetros posicionales con los nombres \$1, \$2, ..., \$n
\item Los argumentos es información que puede ser enviada a la función en forma de una lista de cadenas separadas por espacio.\\ 
\item Los parámetros posicionales son variables, por lo que pueden ser utilizadas como las definidas explicitamente.
\texttt{Ejemplo : function sumar()\{ echo \$((\$1+\$2+\$3)); \} }\\ 
\texttt{Ejemplo de llamada : sumar 2 3 4 }
\end{itemize}
\end{frame}






\begin{frame}
Bibliografía : 
\begin{itemize}
\item man bash
\item Libro Bash Guide for Beginners, de Machtelt Garrels 
\item Libro Advanced Bash-Scripting Guide, de Mendel Cooper
\end{itemize}

Ambos libros están disponibles en varios formatos diferentes en :

Bibliografía suplementaria sugerida : 
\begin{itemize}
\item Libro El Entorno De Programación Unix, de Kernighan, y Rob Pike
\end{itemize}

\end{frame}



\end{document}


